\documentclass{l3doc}
\title{The \texttt{parse\_icc\_tex} module -- A \TeX\ interface for luaicc}
\author{Marcel F. Krüger}
\begin{document}
\maketitle
\begin{documentation}
The \texttt{parse\_icc\_tex} Lua module defines a plain \TeX\ like interface consisting of three commands:

\begin{function}{\LoadProfile}
  \begin{syntax}
    |\LoadProfile| \meta{csname} \Arg{filename}
  \end{syntax}
  \meta{csname} will be defined to represent the profile loaded from \Arg{filename} in other commands. If \meta{csname} is already defined, it will be overwritten.

  The defined control sequence \meta{csname} can not be used on it's own but only in other commands from luaicc.

  A simple example would be
\begin{verbatim}
% \LoadProfile \sRGB {sRGB.icc}
\end{verbatim}
\end{function}

\begin{function}[EXP]{\ProfileInfo}
  \begin{syntax}
    |\ProfileInfo components| \meta{profile}\\
    |\ProfileInfo class| \meta{profile}\\
  \end{syntax}
  The profile provided in \meta{profile} must be a control sequence defined with \cmd\LoadProfile.

  When called with the |components| option, the function expands to the number of components in it's color space. (E.g.\ 3 for RGB spaces, 4 for CMYK spaces, etc.) The number will always be between 1 and 15.

  When called with the |class| option, the expansion is the 4 character tag representing the profile class. The seven options are |scnr| for input device profiles, |mntr| for display device profiles, |prtr| for output device profiles, |spac| for color space profiles, |link| for device link profiles, |abst| for abstract profiles and |nmcl| for named color profiles.

  The number of components for the \cmd\sRGB\ profile loaded by the previous example could e.g.\ be queried with
\begin{verbatim}
% \ProfileInfo components\sRGB
\end{verbatim}
\end{function}

\begin{function}{\ApplyProfile}
  \begin{syntax}
    |\ProfileInfo |[|delim|\meta{delimiter}] [|gamut|\meta{out-of-gamut tag}] [\meta{rendering intent}] [\meta{interpolation space} [|inverse|]] \meta{target profile} \meta{n}\\
    \meta{source profile$_1$} \meta{source color$_1$} \meta{weight$_1$}\\
    \dots\\
    \meta{source profile$_{n-1}$} \meta{source color$_{n-1}$} \meta{weight$_{n-1}$}\\
    \meta{source profile$_n$} \meta{source color$_n$}\\
  \end{syntax}
  (The [] here do indicate optional arguments and not literal |[| and |]| to be written in the source code.)

  To actually convert colors between profiles, \cmd\ApplyProfile\ is used. Beside just converting, it also allows interpolating between colors in different colorspaces.

  It expands to the components of the color separated by \meta{delimiter} (which must be a single token and defaults to spaces if it is not provided) in the profile given by \meta{target profile}.
  If |gamut| is given, the token provided as \meta{out-of-gamut tag} is prepended if the color is outside of the gamut of \meta{target profile}.

  The used rendering intent is given by \meta{rendering intent}. It must be one of \texttt{perceptual}, \texttt{colorimetric}, \texttt{absolute}, or \texttt{saturation}. If it is not provided, the default is unspecified and might change in later versions.

  The number of source colors is given in \meta{n}.
  It must be a positive integer.
  If it is not 1, the source colors and interpolated based on the intergers given as \meta{weight$_i$}.
  The last weight is not explicitly provided but automatically determined such that the sum of all weights is 1000.

  The colorspace the interpolation i done is is selected by the \meta{interpolation space} option. The options are \texttt{lab} for CIELab, \texttt{xyz} for CIEXYZ, \texttt{xyy} for xyY, \texttt{luv} for CIELUV and the cylindrical options \texttt{lch} for CIELCh and \texttt{lchuv} for CIELCh(uv). The default in unspecified and subject to change.

  The two cylindrical spaces can be followed by the \texttt{inverse} keyword to interpolate along the longer instead of the shorter path for the hue component.

  The \meta{source color$_i$}'s are given by space separated components corresponding to the given \meta{source profile$_i$}.
  Except if \meta{source profile$_i$} is a named color profile, then \meta{source color$_i$} is a single braced argument containing a color name.

  As a special case, when \meta{target profile} is a device link profile, then \meta{n} is not given explicitly and implicitly has the value $1$. Also \meta{source profile$_1$} is omitted since it does not apply.

  An example for simple color conversion from the cyan primary in a CMYK space \cmd\myCMYK\ to a RGB space \cmd\sRGB\ with result components separated by commas would be
\begin{verbatim}
% \ApplyProfile delim, \sRGB 1 \myCMYK 1 0 0 0
\end{verbatim}

To mix 10\% of \cmd\sRGB's green with 30\% of \cmd\myCMYK's yellow and 60\% of \cmd\sRGB's red and return the (space separated) result in \cmd\myCMYK, while doing all calculations in CIELUV, the invocation is
\begin{verbatim}
% \ApplyProfile luv \myCMYK 3
% \sRGB     0 1 0   100
%   \myCMYK 0 0 1 0 300
%   \sRGB   1 0 0
\end{verbatim}
\end{function}

\end{documentation}
\end{document}
